\documentclass[11pt,letter,fleqn,english,notitlepage]{article}
\usepackage{a4}
\usepackage{babel}
\usepackage{german}
\usepackage[dvips]{rotating}
\usepackage{umlaute}
\usepackage{amsmath}
\usepackage{amssymb}
\usepackage{natbib}
\usepackage[dvips]{graphics}
\usepackage{setspace}
\usepackage{geometry}
\usepackage{epsfig}
\usepackage[dvips]{color}
\usepackage{eucal}
\usepackage{fancyhdr}
\usepackage{url}
\usepackage{watermark}
%
\lhead[\fancyplain{}{\emph{A X I S E M}}] {\fancyplain{}{\emph{A X I S E M}}}
\chead[\fancyplain{}{}]                 {\fancyplain{}{}}
\rhead[\fancyplain{}{\emph{Tarje Nissen-Meyer}}]       {\fancyplain{}{\emph{Tarje Nissen-Meyer}}}
\lfoot[\fancyplain{}{}]                 {\fancyplain{}{}}
\cfoot[\fancyplain{}{\rightmark}]         {\fancyplain{}{\thepage}}
\rfoot[\fancyplain{} {}]  {\fancyplain{}{}}
%
\title{AXISEM 1.2 Manual}
\author{Tarje Nissen-Meyer, Martin van Driel, Simon Stähler, Kasra Hosseini}
%
\setlength{\topmargin}{-15mm}
\setlength{\textwidth}{165mm}
\setlength{\textheight}{220mm}
\hoffset=-5pt

%%%%%%%%%%%%%%  Newcommands paper3  %%%%%%%%%%%%%%%%%%%%
%
\newcommand{\eq}{\begin{equation}} \newcommand{\en}{\end{equation}}
\newcommand{\eqa}{\begin{eqnarray}} \newcommand{\ena}{\end{eqnarray}}
\newcommand{\subearth}{\raise.15ex\hbox{$\scriptstyle\oplus$}}
\newcommand{\Earth}{\raise.15ex\hbox{$\oplus$}}
\newcommand{\br}{\mbox{${\bf r}$}} \newcommand{\bs}{\mbox{${\bf s}$}}
\newcommand{\bu}{\mbox{${\bf u}$}} \newcommand{\bv}{\mbox{${\bf v}$}}
\newcommand{\bx}{\mbox{${\bf x}$}} \newcommand{\bw}{\mbox{${\bf w}$}}
\newcommand{\bC}{\mbox{${\bf C}$}} \newcommand{\bE}{\mbox{${\bf E}$}}
\newcommand{\bG}{\mbox{${\bf G}$}} \newcommand{\bI}{\mbox{${\bf I}$}}
\newcommand{\bM}{\mbox{${\bf M}$}} \newcommand{\bT}{\mbox{${\bf T}$}}
\newcommand{\bD}{\mbox{${\bf D}$}} \newcommand{\bU}{\mbox{${\bf U}$}}
\newcommand{\bK}{\mbox{${\bf K}$}} \newcommand{\bS}{\mbox{${\bf S}$}}
\newcommand{\bA}{\mbox{${\bf A}$}} \newcommand{\bQ}{\mbox{${\bf Q}$}}
\newcommand{\bF}{\mbox{${\bf F}$}} \newcommand{\bW}{\mbox{${\bf W}$}}
\newcommand{\bP}{\mbox{${\bf P}$}} \newcommand{\bB}{\mbox{${\bf B}$}}
\newcommand{\bR}{\mbox{${\bf R}$}} \newcommand{\bGamma}{\mbox{${\bf \Gamma}$}}
\newcommand{\bX}{\mbox{${\bf X}$}}
\newcommand{\rsubr}{\br_{\rm r}} \newcommand{\rsubs}{\br_{\rm s}}
\newcommand{\bzero}{\mbox{${\bf 0}$}}
\newcommand{\bdelta}{\mbox{\boldmath $\bf \delta$}}
\newcommand{\bPsi}{\mbox{\boldmath $\bf \Psi$}}
\newcommand{\bdel}{\mbox{\boldmath $\bf \nabla$}}
\newcommand{\bNcal}{\mbox{\boldmath $\bf {\mathcal N}$}}
\newcommand{\bFcal}{\mbox{\boldmath $\bf {\mathcal F}$}}
\newcommand{\bDcal}{\mbox{\boldmath $\bf {\mathcal D}$}}
\newcommand{\bsfM}{\mbox{\boldmath $\bf {\mathsf M}$}}
\newcommand{\bsfK}{\mbox{\boldmath $\bf {\mathsf K}$}}
\newcommand{\bsfU}{\mbox{\boldmath $\bf {\mathsf U}$}}
\newcommand{\bsfF}{\mbox{\boldmath $\bf {\mathsf F}$}}
\newcommand{\bsfp}{\mbox{\boldmath $\bf {\mathsf p}$}}
\newcommand{\bsfq}{\mbox{\boldmath $\bf {\mathsf q}$}}
\newcommand{\bsfu}{\mbox{\boldmath $\bf {\mathsf u}$}}
\newcommand{\bsfv}{\mbox{\boldmath $\bf {\mathsf v}$}}
\newcommand{\bsfw}{\mbox{\boldmath $\bf {\mathsf w}$}}
\newcommand{\bsff}{\mbox{\boldmath $\bf {\mathsf f}$}}
\newcommand{\bsfQ}{\mbox{\boldmath $\bf {\mathsf Q}$}}
\newcommand{\bsfW}{\mbox{\boldmath $\bf {\mathsf W}$}}
\newcommand{\bsfB}{\mbox{\boldmath $\bf {\mathsf B}$}}
\newcommand{\bsfI}{\mbox{\boldmath $\bf {\mathsf I}$}}
\newcommand{\bsfJ}{\mbox{\boldmath $\bf {\mathsf J}$}}
\newcommand{\bsfchi}{\mbox{\boldmath $\bf {\mathsf \chi}$}}
\newcommand{\bsfXi}{\mbox{\boldmath $\bf {\mathsf \Xi}$}}
\newcommand{\bsfYcal}{\mbox{\boldmath $\bf {\mathcal Y}$}}

\newcommand{\bsfzero}{\mbox{\boldmath $\bf {\mathsf 0}$}}
\newcommand{\bsfone}{\mbox{\boldmath $\bf {\mathsf 1}$}}
\newcommand{\sP}{\mbox{$\cal P$}} \newcommand{\sR}{\mbox{$\cal R$}}
\newcommand{\sS}{\mbox{$\cal S$}}
\newcommand{\bkh}{\mbox{$\hat{\mbox{${\bf k}$}}$}}
\newcommand{\blh}{\mbox{$\hat{\mbox{${\bf l}$}}$}}
\newcommand{\bnh}{\mbox{$\hat{\mbox{${\bf n}$}}$}}
\newcommand{\bph}{\mbox{$\hat{\mbox{${\bf p}$}}$}}
\newcommand{\brh}{\mbox{$\hat{\mbox{${\bf r}$}}$}}
\newcommand{\bsh}{\mbox{$\hat{\mbox{${\bf s}$}}$}}
\newcommand{\bth}{\mbox{$\hat{\mbox{${\bf t}$}}$}}
\newcommand{\bxh}{\mbox{$\hat{\mbox{${\bf x}$}}$}}
\newcommand{\byh}{\mbox{$\hat{\mbox{${\bf y}$}}$}}
\newcommand{\bzh}{\mbox{$\hat{\mbox{${\bf z}$}}$}}
\newcommand{\bthetah}{\mbox{$\hat{\mbox{\boldmath $\bf \theta$}}$}}
\newcommand{\bphih}{\mbox{$\hat{\mbox{\boldmath $\bf \phi$}}$}}
\newcommand{\bplh}{\mbox{$\hat{\mbox{\boldmath $\bf +$}}$}}
\newcommand{\bmih}{\mbox{$\hat{\mbox{\boldmath $\bf -$}}$}}
\newcommand{\ssR}{\mbox{${\sf R}$}} \newcommand{\sszero}{\mbox{${\sf
0}$}} 

\newcommand{\uto}{\mbox{${\bf u}^{\hspace{-1.6ex}\raisebox{0.5ex}{$\scriptscriptstyle\rightarrow$}}$}} 
\newcommand{\vto}{\mbox{${\bf v}^{\hspace{-1.6ex}\raisebox{0.5ex}{$\scriptscriptstyle\rightarrow$}}$}} 
\newcommand{\vtos}{\mbox{${v}^{\hspace{-1.6ex}\raisebox{0.5ex}{$\scriptscriptstyle\rightarrow$}}$}} 
\newcommand{\Eto}{\mbox{${\bf E}^{\hspace{-1.6ex}\raisebox{1.0ex}{$\scriptscriptstyle\rightarrow$}}$}} 
\newcommand{\Tto}{\mbox{${\bf T}^{\hspace{-1.6ex}\raisebox{1.0ex}{$\scriptscriptstyle\rightarrow$}}$}} 
\newcommand{\Gto}{\mbox{${\bf G}^{\hspace{-1.6ex}\raisebox{1.0ex}{$\scriptscriptstyle\rightarrow$}}$}}
\newcommand{\ruto}{\mbox{${u}^{\hspace{-1.1ex}\raisebox{0.5ex}{$\scriptscriptstyle\rightarrow$}}$}}
\newcommand{\rvto}{\mbox{${v}^{\hspace{-1.1ex}\raisebox{0.5ex}{$\scriptscriptstyle\rightarrow$}}$}}
\newcommand{\rEto}{\mbox{${E}^{\hspace{-1.3ex}\raisebox{1.0ex}{$\scriptscriptstyle\rightarrow$}}$}}
\newcommand{\rTto}{\mbox{${T}^{\hspace{-1.4ex}\raisebox{1.0ex}{$\scriptscriptstyle\rightarrow$}}$}}
\newcommand{\svto}{\mbox{${\sf v}^{\hspace{-1.2ex}\raisebox{0.5ex}{$\scriptscriptstyle\rightarrow$}}$}} 
\newcommand{\sEto}{\mbox{${\sf E}^{\hspace{-1.3ex}\raisebox{1.0ex}{$\scriptscriptstyle\rightarrow$}}$}} 
\newcommand{\bAto}{\mbox{${\bf A}^{\hspace{-1.8ex}\raisebox{1.0ex}{$\scriptscriptstyle\rightarrow$}}$}} 
\newcommand{\Ato}{\mbox{${A}^{\hspace{-1.3ex}\raisebox{1.0ex}{$\scriptscriptstyle\rightarrow$}}$}}


\newcommand{\ufrom}{\mbox{${\bf u}^{\hspace{-1.6ex}\raisebox{0.5ex}{$\scriptscriptstyle\leftarrow$}}$}} 
\newcommand{\vfrom}{\mbox{${\bf v}^{\hspace{-1.6ex}\raisebox{0.5ex}{$\scriptscriptstyle\leftarrow$}}$}} 
\newcommand{\vfroms}{\mbox{${v}^{\hspace{-1.6ex}\raisebox{0.5ex}{$\scriptscriptstyle\leftarrow$}}$}} 
\newcommand{\Efrom}{\mbox{${\bf E}^{\hspace{-1.6ex}\raisebox{1.0ex}{$\scriptscriptstyle\leftarrow$}}$}} 
\newcommand{\Tfrom}{\mbox{${\bf T}^{\hspace{-1.6ex}\raisebox{1.0ex}{$\scriptscriptstyle\leftarrow$}}$}}
\newcommand{\rufrom}{\mbox{${u}^{\hspace{-1.1ex}\raisebox{0.5ex}{$\scriptscriptstyle\leftarrow$}}$}}
\newcommand{\rvfrom}{\mbox{${v}^{\hspace{-1.1ex}\raisebox{0.5ex}{$\scriptscriptstyle\leftarrow$}}$}}
\newcommand{\rEfrom}{\mbox{${E}^{\hspace{-1.3ex}\raisebox{1.0ex}{$\scriptscriptstyle\leftarrow$}}$}}
\newcommand{\rTfrom}{\mbox{${T}^{\hspace{-1.4ex}\raisebox{1.0ex}{$\scriptscriptstyle\leftarrow$}}$}} 
\newcommand{\svfrom}{\mbox{${\sf v}^{\hspace{-1.2ex}\raisebox{0.5ex}{$\scriptscriptstyle\leftarrow$}}$}} 
\newcommand{\sEfrom}{\mbox{${\sf E}^{\hspace{-1.3ex}\raisebox{1.0ex}{$\scriptscriptstyle\leftarrow$}}$}} 
\newcommand{\bAfrom}{\mbox{${\bf A}^{\hspace{-1.8ex}\raisebox{1.0ex}{$\scriptscriptstyle\leftarrow$}}$}}
\newcommand{\Afrom}{\mbox{${A}^{\hspace{-1.3ex}\raisebox{1.0ex}{$\scriptscriptstyle\leftarrow$}}$}} 

\newcommand{\bAboth}{\mbox{${\bf A}^{\hspace{-1.7ex}\raisebox{1.0ex}{$\scriptscriptstyle\leftrightarrow$}}$}}
\newcommand{\Aboth}{\mbox{${A}^{\hspace{-1.1ex}\raisebox{1.0ex}{$\scriptscriptstyle\leftrightarrow$}}$}}
%

%
\begin{document}
%
\pagestyle{fancy}
\thispagestyle{empty}
%
\begin{center}
{\LARGE {\sc AXISEM 1.2}}
\vspace{1.cm}\\
{\large 
Tarje Nissen-Meyer\textsuperscript{3)},
Martin van Driel\textsuperscript{2)},
Kasra Hosseini\textsuperscript{1)}
Simon St\"{a}hler\textsuperscript{1)}} \\
{\small \textsuperscript{1)} LMU Munich, \textsuperscript{2)} ETH Zurich, 
\textsuperscript{3)} Oxford University \\
\vspace*{0.2cm}
\textit{tarje@alumni.princeton.edu} \hspace*{0.75cm}
\vspace*{0.5cm}\\ 
\today}
\end{center}
\thiswatermark{\begin{picture}(0,0) \put(-20,-190){\hspace{1.5cm}\includegraphics[width=0.9\linewidth]
{snap_cover_red.eps}}\end{picture}}

\vspace*{1.6cm}
\noindent {\sc AXISEM} is a \textbf{parallel spectral-element method} to solve
3D wave propagation in a sphere with axisymmetric or spherically symmetric
visco-elastic, acoustic, anisotropic structures. Such media allow the
computational domain to be collapsed to a 2D disk, where the third, azimuthal
dimension is solved analytically on-the-fly posteriori. This leads to extreme
speedup by many orders of magnitude with respect to methods that discretize the
3D domain, and enables a full coverage of the seismic body- and surface wave
frequency spectrum between 0.001-1Hz.  The time-domain code delivers full
spatio-temporal wavefields that can be stored on disk and transformed to
frequency domain. Due to the dimensional reduction, global wave propagation at
typical seismic of \textbf{periods down to 5 seconds can be tackled on
laptops}, and at 1Hz on moderate clusters.

%Exploitation 
%of moment-tensor source and single-force radiation patterns allow the \textbf{computational domain 
%to be collapsed to a 2D semi-disk}, and the azimuthal third dimension is computed analytically.
%For each full earthquake moment tensor (single-force vector), 4 (2) simulations are undertaken that account for 
%tensor (vector) elements separately and are summed subsequently.
The Fortran 90 code is divided into a \textbf{Mesher}, a \textbf{Solver}
utilizing the message-passing interface (MPI) for communication between
separate domains, and \textbf{extensive post processing} for ease of
visualization.
%Radiation pattern symmetries require all sources 
%to be located along the axis, and any lateral heterogeneities are translated into a 2.5-dimensional
%ring-like structure.
The essential raison-d'\^{e}tre of this method is the \textbf{efficient
calculation of seismograms, wavefield movies, and those wavefields that underly
sensitivity kernels} to allow for tomographic inversions of any portion of a
seismogram at any relevant frequency. \\
%We have added methodological foundations directly extracted from the references.
%Eventually, we will add the sensitivity kernel software as well. \\

\noindent Provisional portal for this code at ETH Zurich:
\url{www.axisem.info}
%Please contact Tarje (\url{tarjen@ethz.ch}) for login information.

\section{Authors, contributors \& copyright}
%
\noindent \textbf{Principal authors:} Tarje Nissen-Meyer, Alexandre Fournier, Martin van Driel\\
\noindent \textbf{Contact:} Tarje Nissen-Meyer (\url{tarje@alumni.princeton.edu})\\
\noindent \textbf{Contributions:}
 J.-P. Ampuero, E. Chaljub, A. Colombi, F. A. Dahlen, S. Hempel, K. Hosseini, D. Komatitsch,
 G. Nolet, K. Sigloch, S. St\"{a}hler, J. Tromp.\\
\noindent \textbf{Research funding:} Princeton University, NSF (USA), SNF (Switzerland), 
the QUEST initial training network and ETH Zurich.\\
\noindent \textbf{Guarantees:} No guarantee whatsoever is given for this
software under any circumstances. It may be used freely for academic purposes
under the GNU license. Commercial use must be discussed with the authors prior
to usage.\\

\noindent \copyright  \hspace*{0.1cm} 
2007-2013 Tarje Nissen-Meyer (\url{tarje@alumni.princeton.edu}), ETH Zurich
% TODO GPL?

\newpage
\tableofcontents
\newpage

\section{Preliminaries}

\subsection{Software and hardware requirements}

\textbf{Essential requirements:} 
\begin{itemize}
\item \textit{Compilers:} Fortran 90 compiler (tested on ifort, gfortran, portland)
\item \textit{Libraries:} MPI (tested on OpenMPI, mpich)
\item \textit{Systems:} Unix-based OS (tested on 32/64 Bit Linux clusters, Dell Nehalem, OS X.4-7, Cray XT4,
Cray XE6 and XK7)
\end{itemize}

\textbf{Optional embedded software/libraries:} netcdf, fftw3, python, taup, gnuplot

\textbf{Optional processing software:} xmgr, matlab, paraview, google earth, obspy
\subsection{References}

\noindent \textbf{Directly dealing with this code:}\vspace*{0.2cm}

(1) Tarje Nissen-Meyer, F. A. Dahlen, A Fournier (2007),
\textit{Spherical-earth Fr\'{e}chet sensitivity kernels},        
Geophysical Journal International 168(3),1051-1066. 
doi:10.1111/j.1365-246X.2006.03123.x                \\
                                                        
(2) Tarje Nissen-Meyer, A Fournier, F. A. Dahlen (2007), 
\textit{A two-dimensional spectral-element method for
spherical-earth seismograms-I. Moment-tensor source}, 
Geophysical Journal International 168(3), 1067-1092. 
doi:10.1111/j.1365-246X.2006.03121.x                 \\
                                                       
(3) Tarje Nissen-Meyer, A Fournier, F. A. Dahlen (2008),  
\textit{A two-dimensional spectral-element method for   
spherical-earth seismograms - II. Waves in solid-fluid media},
Geophysical Journal International, 174(3), 873-888.
doi:10.1111/j.1365-246X.2008.03813.x\\

(4) Tarje Nissen-Meyer (2007),
\textit{Full-wave seismic sensitivity in a spherical Earth},
Ph.D. thesis, Princeton University
(This includes refs (1)-(3) and more details.)\\

(5) Jean-Paul Ampuero, Tarje Nissen-Meyer (2011),
\textit{High-order conservative time schemes in spectral-element methods 
for seismic wave propagation.}, To be submitted to Geophys. J. Int.\\

%\noindent \textbf{NOTE:} Most important sections of these papers are in the Theory section below.\\

\noindent \textbf{Other references:}\vspace*{0.2cm}

(6) Deville, M. O., Fischer, P. F., Mund, E. H. (2002), 
\textit{High-Order Methods for Incompressible Fluid Flow}, 
Vol. 2, Cambridge monographs on Sppl. \& Comp. Math., Cambridge University Press.\\

(7) Tufo, H. M., Fischer, P. F. (2001), \textit{Fast Parallel Direct Solvers For Coarse Grid Problems}, 
61, 151-177, J. Par. and Dist. Comput.\\

(8) Bernardi, C., Dauge, M., Maday, Y. (1999), \textit{Spectral Methods for Axisymmetric Domains}, 
Vol. 3, Series in Appl. Math., Gauthier-Villars, Paris.\\

(9) Chaljub, E. (2000), \textit{Mod{\'{e}}lisation num{\'{e}}rique de la 
propagation d'ondes sismiques en g{\'{e}}om{\'{e}}trie sph{\'{e}}rique:
Application {\`{a}} la sismologie globale}, 
Ph.D. thesis, Universit{\'{e}} de Paris 7.\\

(10) Komatitsch D., Tromp, J. (2002), \textit{Spectral-element simulations of
global seismic wave propagation---I. Validation},
149, 390-412, Geophys. J. Int.

\section{Overview}

\subsection{Minimalist approach}
This is the step-by-step, blackbox procedure, i.e. running a workflow from raw source code to analyzing 
seismograms and wavefield movies upon pre-set parameters.\\

\noindent \textit{Basic requirements}: gfortran and MPI compiled for gfortran.\\
\textit{Processing requirements (not obligatory)}: gnuplot, taup\\
\textit{Visualization tools (suggested):} paraview, matlab, googleearth\\

\noindent Start from within the {\tt AXISEM} directory:
\begin{enumerate}
\item {\tt cd MESHER}
\item {\tt ./makemake.pl} $\Rightarrow$ creates Mesher Makefile.
%\item {\tt make} $\Rightarrow$ make sure {\tt xmesh} exists.
\item {\tt ./submit.csh} $\Rightarrow$ Check {\tt OUTPUT}.
\item Wait for ``{\tt ....DONE WITH MESHER}'' to appear in {\tt OUTPUT}.
\item {\tt ./movemesh.csh TEST\_MESH} $\Rightarrow$ moves mesh files to {\tt ../SOLVER/MESHES/TEST\_MESH}.
\item {\tt cd ../SOLVER}
\item In \verb|inparam_basic| set \verb|MESHNAME| to the meshname from above(\verb|TEST_MESH|)
\item {\tt ./makemake.pl} $\Rightarrow$ creates Solver Makefile.
\item {\tt ./submit.csh TEST\_SOLVER}  $\Rightarrow$ compiles and runs the code
\item {\tt cd TEST\_SOLVER} $\Rightarrow$ Check {\tt OUTPUT\_TEST\_SOLVER}.
\item \verb|tail -f OUTPUT_TEST_SOLVER| $\Rightarrow$ Wait for ``{\tt  PROGRAM
        axisem FINISHED}'' to appear in \\ {\tt OUTPUT\_TEST\_SOLVER}.
\item {\tt ./post\_processing.csh}
\item {\tt cd Data\_Postprocessing} 
\item {\tt googleearth}, open {\tt googleearth\_src\_rec\_processed.kml}, click
        earthquake (info), receivers (seismograms).
\item {\tt matlab}, run {\tt plot\_record\_section.m}, plotting all components of displacement seismograms.
% \item {\tt cd SNAPS; paraview}, load snaps for 3D wavefield movie.
% snaps are deacticated in defaulf...

\end{enumerate}
Once this has been succesfully completed, steps 2. and 8. can be omitted. If
the Solver is re-run with different parameters but the
same mesh, you may start at step 9. Changing mesh input is done between 2. and
3., changing solver input between 8. and 9.,
changing post-processing input between 11. and 12. Using a new mesh requires
recompilation of the solver (done automatically in step 9.). If post
processing parameters are changed, also change the post processing directory or
delete the old one.

\subsection{General remarks on the codes}

\textbf{Overview.} 

% TODO these 2 block are in large parts repetition of the introduction
AXISEM is a parallel spectral-element method of the Gauss-Lobatto-Legendre type
to solve the 3D solid-fluid equations of motion in a spherically symmetric
sphere. The Fortran 90 source code is divided into a Mesher and a Solver
utilizing the message-passing interface (MPI) for communication between
separate domains. Due to symmetries in the earthquake radiation patterns, the
3D wavefield of an earthquake moment tensor is reconstructed from 4 separate
simulations within a 2D computational domain. This D-shaped domain
includes a non-physical boundary with singularities at the symmetry axis which
is accomodated by Gauss-Lobatto-Jacobi discretization and l'Hospital's rule.
Due to this symmetry, earthquake
sources need to be located along this axis, and any lateral heterogeneities in
the background model would be seen in an effectively ring-like structure.\\

The essential raison-d'\^{e}tre of this method is the efficient calculation of
seismograms, wavefield movies, and specifically wavefields that constitute the
crux for sensitivity kernels to allow for tomographic inversions of any portion
of a seismogram at any relevant frequency. The two main parts are the mesher to
construct a 2D domain for a given number of processors, resolution and input
background model, and the solver which conducts the temporal evolution of the
system explicitely at arbitrary spatial and up to sixth-order temporal
accuracy.\\

\textbf{Code structure \& coding philosophy}

\begin{itemize}
    \item The $>$40.000 lines of code are very \textbf{modular} in nature, such
    that vast parts of the code shall never need to be seen by anyone wishing
    to change/add something. This is particularly the case for the intricate
    geometrical mapping relations, quadrature, spectral operations,
    interpolation, index-mapping in the mesher, and some of the I/O.
    
    \item Check {\tt main.f90} in either code for an overview of the tasks.
    Most important routines are well documented (at least in the Solver case!)

    %  TODO Is this of relevance for average users?
    \item The codes allow for single or double precision (specified in {\tt
    global\_parameters.f90}), but in any case, crucial geometric operations are
    done in double, whereas large wavefield arrays in single can reduce
    computational cost notably if chosen.

    \item The Mesher allocates memory dynamically, whereas the Solver includes
    the major array sizes from the mesher into a header ({\tt mesh\_params.h}).
    This means that each new mesh requires recompilation of the Solver. 

    \item Input options are few and physical: The aim is to streamline as much
    as possible and leave only those options that relate to practical choices.

    \item All variables and constants that are needed across modules are
    defined in files named {\tt data\_xxx.f90}; global constants in {\tt
    global\_parameters.f90}.

    \item Changes in any of the input files do NOT require recompilation,
    changes in {\tt *.f90} or {\tt *.h} files DO require recompilation. 

    \item Standard output is written to files called {\tt OUTPUT} (mesher) and
    {\tt OUTPUT\_<solverdir>} (solver), which contain all relevant information
    regarding either run. 

    \item Optimization: The solver runs fully on unit-stride cache access and
    unrolled loops for tensor products as well as asynchronous message passing.

    \item Everything related to MPI and parallel worlds is confined to the
    module {\tt commpi.f90}. In order to completely debunk message passing, one
    simply needs to remove {\tt commpi.f90}, and change a few lines in {\tt
    commun.f90} (as indicated there), then restart the {\tt makemake.pl; make
    clean; make} workflow. 
    % TODO this has not been tested in a while

    \item Object-oriented fortran features have been ommitted almost entirely
    due to their severe negative impact on performance.
\end{itemize}

\subsection{General workflow}
The generic workflow for both Mesher and Solver is as follows:
\begin{enumerate}
    \item {\tt ./makemake.pl <arguments>}
    \item {\tt make clean; make}
    \item Edit input parameters
    \item {\tt ./submit.csh <arguments>}
\end{enumerate}

\noindent {\tt makemake.pl } generates the respective Makefiles. Argument \#1 options 
(check available options with {\tt ./makemake.pl -h}) include:
{\tt gfortran} and {\tt ifort}, argument \#2 is optional and only defined for
{\tt debug}. Best is to try some options and check the Makefile to make sure it suits you.\\

\noindent \textbf{Input parameters: } Input ({\tt vi inparam\_mesh})
to the Mesher is simple, you should in most cases (of conventional earth models) 
not need to edit more than the first 3 parameters (earth model, seismic period, and number of processors). 
Input to the Solver is attempted to be streamlined to the most primitive basis of settings as well. General 
input parameters are set in {\tt inparam}, source and receiver parameters in separate files (specified 
in {\tt inparam}). The information from the mesh is confined to the header file {\tt mesh\_params.h} .\\

\noindent {\tt submit.csh} are scripts to submit the jobs, check details with
{\tt ./submit.csh -h} for both Mesher and Solver. In both cases, one can
choose the queuing system (so far lsf, slurm and torque/maui) via arguments. In the
Solver case, the source radiation type can also be specified via arguments.
Note that in the case of a full moment tensor, 4 jobs will be submitted.
% TODO: not implemented: , and in the case of a full force vector 2 (). 
The script for the solver performs a variety
of crucial operations for the simulations to succeed, act with sensitive care
in case you really need to edit this script! \\

\noindent Upon successful completion of either job type, several
post-processing options are available to indulge in. The mesher contains
several vtk files to check the mesh via paraview, the solver contains a
comprehensive and important series of operations invoked via {\tt
./post\_processing.csh } from the directory of the run. This includes summation
over individual moment-tensor element responses, convolution with source time
functions, rotation to actual source-receiver geometry, google-earth plots,
seismograms in ascii and graphical formats, traveltime tables, 3D wave
propagation snapshot movies, and a matlab script for plotting seismogram record
sections.

\subsection{Algorithmic \& technical features}
\begin{itemize}
    \item Global numbering scheme (P. Fischer \& H. Tufo)
    \item High-order symplectic time integration schemes (co-developed with Jean-Paul Ampuero)
    % TODO: these are not well tested and give different results then newmark atm.
    \item Unit-stride cache access, unrolled matrix products (Deville, Fischer, Mund)
    \item Asynchronous (non-blocking) message passing (Tufo)
    \item Output in VTK cell geometry
    \item netcdf, xdmf file formats
    \item python interface 
\end{itemize}

\subsection{File formats}
???

\subsection{Automated testing}

This section is mainly useful for the develpers or those want to change/add/remove some parts of the code and compare the new changes with the reference solutions.
For the reference solutions, 5 different tests have been designed and 
the results from \textit{yspec} [Al-Attar \& Woodhouse (2008)] and AXISEM have been included in the \textit{TESTING/automatic} directory.
The whole procedure (running the code, compare the results and plot) is automatic with the least user intervention: \\

Start from within the {\tt AXISEM} directory:
 \begin{enumerate}
 \itemsep0em
 \item {\tt cd TESTING}
 \item {\tt python test\_axisem.py}
 \end{enumerate}

Enter test number(s) and this is all you should do! \\

\noindent As an example, we want to run the \textit{test\_axisem.py} for test number 5. (Figure~\ref{test_axisem})

\begin{figure*}[htb]
\begin{center}
\includegraphics[scale=0.4]{PYAXI/test_axisem.eps}
\caption{\textit{Screenshot while running test\_axisem.py}}
%\caption{\textit{}}
\end{center}
\label{test_axisem}
\end{figure*}

\noindent At the end, it plots three figures (one for each channel) 
in which the new AXISEM waveforms are compared with both the original ones and \textit{yspec} results for the same simulation.
%In Figure~\ref{channel_Z}, only the Z channel has been plotted.
It should be noted that these tests are designed for rough comparison purposes (in terms of the pulse shape and sign) and they should not be considered as a detailed benchmark with respect to YSPEC. \\

\begin{figure*}[htb]
\begin{center}
\includegraphics[scale=0.3]{PYAXI/record_section_Z.eps}
\caption{\textit{Comparing new AXISEM results with the reference solution and \textit{yspec} waveforms. (Z channel)}}
\end{center}
\label{channel_Z}
\end{figure*}

\noindent Another output is the \textit{l2} misfit (Figure~\ref{l2_misfit}) between the new and reference data 
in which the traces can be compared in a quantitative sense.

\begin{figure*}[htb]
\begin{center}
\includegraphics[width=0.8\textwidth]{PYAXI/l2_misfit.eps}
\caption{\textit{l2 misfit between the new and reference data.}}
\end{center}
\label{l2_misfit}
\end{figure*}


% \input{virtual_box_tutorial}
% seperate document by now

\input{pyaxi_input}

\newpage
\section{Mesher}
Constructs the spherical 2-D semi-disk of spectral elements based on a given
background model, a target resolution as defined by the (dominant) source
period, the number of processors, and the polynomail order of the spectral
elements. 
% TODO true, but who cares?
This mesher is NOT well documented and rather involved regarding
index mapping, hence subject to future streamlining efforts. Please let us know
if you intend to add comments or change code to join forces. It invokes a
series of dingy, multiply layered bookkeeping index arrays that are not easy to
follow (see mind-twisting highlights in {\tt parallelization.f90} and {\tt
pdb.f90}). Again, we welcome improvements here but want to caution users to
change anything if not entirely sure about all dependencies.\\

\noindent NOTE: The mesher is serial but constructs the parallel mesh database 
needed in the solver (in {\tt pdb.f90}), easily down to periods of about 
3 seconds on typical shared RAMs ($\sim$ 2GB) and to 1Hz on ($\sim$ 14GB). 
For higher resolutions on such chips, reordering of global variables and/or complete 
parallelization of the Mesher is a mundane task for $t>{\rm today}$.
%  TODO ???

\subsection{Installation and compilation}
Use {\tt ./makemake.pl <arguments>} to create the Makefile when opening for the
first time or adding modules.  Invoking {\tt ./makemake.pl -h} provides options
for different compilers. Always check the resultant {\tt Makefile} to make sure
it rocks, compiler- and flag-wise. Compile via {\tt make clean; make}. The
executable is called {\tt xmesh}.  Changing code without adding modules does
not require to recreate the Makefile, only recompilation.

\subsection{Input}
\begin{figure*}[htb]
\begin{center}
\includegraphics[scale=0.6]{inparam_mesh.eps}
\caption{\textit{{\tt inparam\_mesh}: defines all relevant parameters, mostly self-explanatory. }}
\end{center}
\end{figure*}

\noindent \textit{Basic set of parameters} to be edited:
% TODO: parameter names as in the input files
% TODO: new switch: dump_vtk
\begin{itemize}
    \item \textbf{background models}: Be sure that the string defined here exists in 
    {\tt background\_models.f90}. Adding new background models is explained further below.
    
    \item \textbf{dominant period:} This is the seismic source period at which dominant parts 
    of the spectrum are propagated. Note that this is different from other codes in which the 
    maximal frequency may be specified. 
    In most applications, the solver should be run using a Dirac 
    delta function and then convolved a posteriori with the source-time function (STF). In that 
    case the pure simulation results contain high-frequency numerical noise beyond the mesh 
    resolution, but the convolution eradicates those (even better than if a non-white spectrum 
    STF is inserted as the source, since the spatial point source also introduces aliasing, which 
    can only be taken care of via the above-mentioned posteriori convolution.
    % TODO: sure? should this not be a linear effect? notes on stf seems a bit
    % misplaced here, more relevant for SOLVER then MESHER
    
    \item \textbf{number of processors}: needs to be 1, 2, 6 or a multiple of
    4. To get a suggestion for optimal decomposition, run the Mesher with #proc
    smaller or equal 4 and check the OUTPUT file. Then change the number in the
    input file and rerun the mesher.

    \item \textbf{coarsening levels}: Keep this at 3 if applied to any typical earth models
    (2 for 'light' models without crust), as this number reflects the overall
    change in velocity between the surface and the deep interior.  if the total
    global variation in wave speed is significantly different, other options
    may be convenient here. This applies to other spheroidals, homogeneous or
    simply layered models.
\end{itemize}

\newpage
\noindent \textit{Advanced set of parameters} to be edited:

\begin{itemize}
    \item \textbf{resolve inner shear wave:} This should always be set to true.
    If false, then the inner core is assumed fluid but the saved CPU cost is negligible.
    
    \item \textbf{polynomial order}: The polynomial order $N$ of the
    Gauss-Lobatto-Legendre (GLL) basis within elements is often said to be
    optimal at 4. However, this can be freely chosen, specifically if
    convergence tests are needed, and Ref (5) suggests that higher polynomial
    orders may be more cost-effective, depending on the constraints imposed by
    the element mesh (e.g., thin crustal layers to be accomodated). The
    ``optimal'' judgment stems from the fact that the minimal spacing of GLL
    points depends on $1/N^2$, that is, the spectral convergence is
    counteracted by a possibly quadratic decrease in the critical time step.
    NB: 'Coarse grained' memory variables for efficient attenuation in the solver
    are only implemented for polynomial order 4.
    
    \item \textbf{number of elements per wavelength}: This will be used to
    compute the largest allowable grid spacing based on the dominant source
    period and is intimately tied to the polynomial order $N$. See ref (5) for
    a comprehensive analysis on how to choose the parameters if necessary.  If
    the seismograms contain unreasonable noise (usually high-frequency tails),
    are inaccurate or otherwise different than a reference solution or data,
    try to increase this value. This will result in a denser mesh at higher
    simulation cost (possibly more processors), but deliver more accurate
    results. The absolute minimum number of grid points per maximum wavelength
    for S waves is about 4.5 (i.e., this parameter times $N$). See ref (3) and
    (5) for details.
    
    \item \textbf{Courant number}: Defines the stability criterion to choose
    the critical time step.  0.6 is the heuristically determined maximal value
    for any realistic applications. If the solver explodes, decrease this
    number. Note that dispersion errors increase with propagation distances
    (counted in wavelengths), and for large distances one should  choose a
    smaller time step than the critical one (which is the one suggested by the
    output of the mesher in {\tt mesh\_params.h}).
    
    \item \textbf{outer radius}: This is the same for IASP91 and PREM, but
    should be changed accordingly if other models necessitate otherwise, or of
    course if other spheroidals are considered. 
    
    \item \textbf{mesh info files}: These files become extravagantly large for
    high-resolution meshes and are not needed by the solver, nor by any of the
    current post-processing tools and as such should be kept .false. unless
    significant changes or issues within the mesher are emerging.
    
    \item \textbf{mesh info screen}: This is extra output containing some more
    details on the meshing process, but equally to the large files unnecessary
    for any normal meshing procedure.
    
    \item \textbf{dump directory:} The directory where most of the output is
    written. The posteriori {\tt movemesh} script moves all those files to the
    permanent mesh location.
    
\end{itemize}


\subsection{Running the mesher}
Job submission is simple:\\
{\tt ./submit.csh <optional queuing system>},\\
where {\tt <optional queuing system>} can be {\tt lsf, torque, slurm,
slurmlocal} at this point. Check {\tt ./submit.csh -h} for options.

\subsection{Changing the background model}
This is traditionally the bottleneck of seismic code flexibility (in my very
biased view)... so we've given considerable effort to making this task as
primitive and decoupled from the rest of the code as possible. In principle,
two options for new earth models exist: \\ (1) Parameterized, polynomial,
piecewise continuous representation of wavespeeds $v(r)$,\\ (2) Discrete radial
values of $v(r_i)$.\\

\noindent In principle, this should be entirely general such that you can dream
up multiple fluid layers, discretely layered earths or other planets,
ridiculous velocity variations, however we have not completed the fully fluid
sphere just yet. As soon as your new model affects the overall change in
velocity between surface and maximal value at depth, you will have to change
the number of expected coarsening levels in {\tt inparam\_mesh} as indicated
there. Also, the innermost layer needs to be thick enough to accommodate the
inner cube of the mesh. \\

\noindent Two modules will need to be amended: {\tt model\_discontinuities.f90}
and {\tt background\_models.f90}.  Note that {\tt background\_models.f90} is
copied to the Solver by {\tt movemesh} such that additions will be
automatically accounted for in the Solver. 

\subsection{Anisotropy}
% TODO

\subsection{Attenuation}
% TODO

\subsection{Computational aspects}
The simulations should be done within seconds to minutes for meshes above 5
seconds.  Note that doubling the resolution (half the period) results in a mesh
that is 4 times larger and has about half the time step, i.e. the solver takes
double the time at 4 times more processors. Increasing the number of processors
while keeping resolution constant will slow the mesher insignificantly, but in
most cases speed up the solver substantially, but not linearly (the more
message passing, the slower).  This mesher is completely serial but constructs
the parallel mesh database needed in the solver (in pdb.f90), easily down to
periods of about 3 seconds on typical memory chips ($\sim$ 2GB), and down to 1
second on larger shared memories (e.g. 16GB).
% TODO: this is a repetition

\begin{table}[h]
\begin{tabular}{ccc}
period & max  \\
 T[s]  & nproc & memory [GB] \\
\hline
 50.0 &    8  &  $< 1$  \\
 25.0 &   12  &  $< 1$  \\
 20.0 &   16  &  $< 1$  \\
  9.0 &   32  &  $< 1$  \\
  4.4 &   64  &  $< 1$  \\
  2.2 &  128  &  $ 3.5 $ \\
  1.1 &  256  &  $13.0 $ \\
\end{tabular}
\end{table}

\subsection{Code structure and components}
Check {\tt main.f90} for the general code structure. Meshing of the spheroidal
is confined to the module {\tt discont\_meshing.f90}. The inner cube contains a
coordinate transformation (see Theory section), which is done in {\tt
meshgen.f90}. The domain decomposition is accomodated by {\tt
parallelization.f90}, and the parallel database generated in {\tt pdb.f90}. You
may wish to avoid diving into these murky modules unless your curiosity exceeds
that of an average coding geek, or you encounter an issue.\\

\noindent\textbf{Domain decomposition }\\
In {\tt parallelization.f90} we have an exactly load-balanced colatitudinal
cake-piece decomposition for the outer spherical part of the mesh. The central,
rectangular part is trickier. We employ a method that is based on the
following principles:

\begin{itemize}
    \item exact load balancing 
    \item maximally 2 neighboring domains, i.e. each domain touches the axis
    \item at least two elements thickness
    % TODO: isn't this three?
    \item spherical cake-piece decomposition as 'boundary condition'
\end{itemize}

% TODO: note on new decomposition method

%Upon this, we construct the decomposition by approximating polynomials $s(z)$
%of varying degrees. For the time being, this is only implemented up to 
%16 processors, but no hard limitation. See the Theory section in this manual for details.

\subsection{Output}
\underline{Needed by solver} \\
{\tt mesh\_params.h}\\
{\tt meshdb.dat0000,...,meshdb.dat00$<$nproc-1$>$}\\
%{\tt unrolled\_loops.f90}\\
{\tt background\_models.f90}\\

\noindent\underline{NOT needed by solver }\\
Everything else (mostly in directory {\tt Diags/}) including various global 
arrays such as valence.\\

\noindent {\tt mesh\_params.h}: \\
This is a header file that contains all the static array sizes for the solver
such that it does not need to allocate (a lot of) memory dynamically.
Specifically, it documents all revelant mesh input parameters commented at the
top, and some output information such as time step at the bottom.  The
inclusion of this header into the solver requires recompilation for each new
mesh. This header includes mesh sizes and is identical for each processor due
to the exact load balancing.  The last line is added by {\tt movemesh} and
indicates the location of the mesh database. If you move the mesh, make sure to
change this line accordingly!!  \\

\noindent {\tt meshdb.dat00*}: \\
These are the complete databases for each processor of the solver simulations.
They contain the skeleton mesh, regions, boundary information (axis, free
surface, solid-fluid boundary, processor boundary), and basically everything
needed for a full simulation except for the source-receiver specification which
is an input for the solver. \\

\noindent After these two crucial output file types are written, the mesher
reloads them as a simple test on some of their dependencies. More comprehensive
tests are performed in the solver. \\

\noindent\textbf{\underline{Prepare files to be used in solver}}\\
Use the script {\tt movemesh} to transfer all relevant files to the directory ../SOLVER/MESHES. 
% TODO  I think this is ouddated. double check
%Using this script is crucial since it adds a line to the {\tt mesh\_params.h}
%header file to indicate the location of the large mesh databases.

\subsection{Mesher Post-processing/quality control }
\begin{figure*}[htb]
\begin{center}\label{fig:mesh_vp}
\includegraphics[scale=0.45]{mesh_vp_fig.eps}
\caption{\textit{The elemental mesh (blue lines) for IASP91 at 20 seconds superimposed on the $v_p$ velocity. 
The plot is derived straight from the file {\tt mesh\_vp.vtk} produced by the mesher. Zoom sections of the 
central region and crust/upper mantle are added to highlight the topological features.}}
\end{center}
\end{figure*}
%
\noindent Generic output includes various {\tt *.vtk} files on the elemental
mesh with media properties (see Fig.~\ref{fig:mesh_vp}) as well as a host of
other plots with respect to Courant number, points per wavelength, period, time
step etc, and domain decomposition.  All of these files are moved to the
permanent mesh location via {\tt movemesh} and can be easily viewed with {\tt
paraview}. Turn on the \textit{surface with edges} and \textit{data} features
to visualize the element mesh and medium properties, respectively.

\subsection{Cautions}
Fully operational in single or double precision, so please 
\textbf{avoid using the -r8 (enforced double precision) flag}, 
since the output expects single precision for certain arrays (especially in vtk).

\subsection{Issues}
The hmin/hmax vtk output files have strangely large values, but the period and dt ones are fine.

\subsection{Mesher To Do}
General cleanup, documentation, parallelization, domain decomposition in the central region for 
more than 16 processors, netcdf database.

\newpage
\section{Solver}
Picks up the header from the mesher into its compilation to solve 
the elastodynamic 3-D equations of motion for spherically symmetric 
earth models in a 2-D computational domain using message-passing for the 
parallelization. The full seismic moment-tensor $M_{ij}$ is accounted for by four
separate simulations: two monopole, one dipole, and one quadrupole 
simulation have to be done to sum up to the 6 independent elements of $M_{ij}$.
The software has the additional feature of computing and 
saving entire wavefields that form the basis of full-wave based, 
multiple-frequency sensitivity kernels. This output is included in the 
solver, but the calculation of kernels needs to be done elsewhere. 
Detailed information on the various components of the code are included as 
comments to the respective modules and routines. 
Here, we give a simple overview over the main parts. 
\newpage
\subsection{Installation and compilation}
\begin{enumerate}
\item If first used or new modules are added to the Solver code,
use {\tt ./makemake.csh } to create a new Makefile, see {\tt ./perlmakemake.pl -h} for 
flag options on different compilers. {\tt ./makemake.csh} checks for the
executable ncdump to determine whether your system supports netcdf. If
so, it includes netcdf library and include paths in the Makefile
(please check if they make sense!), and links the module that works
via netcdf in the solver. Still one can dump the output in other
formats, but the module including netcdf is simply used. In case
netcdf is not found, a ghost module is created instead to ensure
compilation. This will disallow usage of netcdf, quite obviously.
\item Copy {\tt mesh\_params.h} from the desired mesh into the source code directory: 
It contains a last line that links to the location of the database such that 
these possibly very large files do not need to be copied. 
\item Compile: {\tt make clean; make}
\item Make sure that executable {\tt xsem} exists.
\end{enumerate}

%\subsection{Pre-processing}
%The {\tt UTILS} directory contains some utilities that may be useful in 
%choosing the right input parameters. 

\subsection{Input}

\textit{Automatically taken from mesher:} \\
{\tt mesh\_params.h}: header including array sizes necessary for the compilation\\
{\tt meshdb.dat00\*}: mesh databases for each processor\\
{\tt unrolled\_loops.f90}: unrolled loops and unit-stride cache access for given polynomial order\\
{\tt background\_models.f90}: all background models. Needs to be the same as in mesher\\

\noindent Any changes to Solver code, background model, mesh, polynomial order etc. require 
recompilation. The above input files are fully automatic and need not be changed for all intents 
and purposes of the Solver.\\

\noindent \textit{Input files for the Solver:} \\
\noindent {\tt inparam, sourceparams.dat, receivers.dat}\\
These 3 solver input files may be changed without recompilation of the solver 
source code. We will describe these files at length in the following. PLEASE do not shy away from the 
length of this section.... the input is fairly easy to understand and manipulate, 
we simply intend to comment as much as possible on all options!\\

\begin{figure*}[htb]
\begin{center}
\includegraphics[scale=0.6]{inparam.eps}
\caption{\textit{{\tt inparam}: defines all relevant parameters, mostly self-explanatory. }}
\end{center}
\end{figure*}
%
\noindent {\tt inparam}:
\begin{itemize}
\item \textbf{number of simulations}: Keep this at 1 for the time being, this is not completed yet. 
If you wish to simulate full moment tensors or forces at once, do that via {\tt submit.csh}.
\item \textbf{seismogram length:} Useful to check this via a ray tracer (taup) before choosing, the overall 
CPU cost scales linearly with this length.
\item \textbf{time step}: If you keep this at 0.0, the code chooses the critical time step as specified in
{\tt mesh\_params.h}.
We suggest to choose a lower value than this critical value, at best some  
integer multiplicative of the seismogram sampling specified further below. The CPU cost 
scales inversely linear with the time step.
\item \textbf{time scheme}: see {\tt time\_evol\_wave.f90} for details. 
{\tt newmark2} is the traditional 2nd-order 
scheme, {\tt symplec4} is a PEFRL scheme of 4th order, about 2.5 more CPU time 
required than the Newmark scheme but significantly more accurate. See ref(5) for a
detailed analysis and rules of thumb for choosing the optimal time scheme. 
Generally, if the setup requires seismograms at more than 100 wavelengths 
in distance, we suggest using the symplectic scheme. The memory requirements 
for the symplectic scheme are in fact *lower* due to avoiding the acceleration
as a full wavefield. See the section on time schemes further below on more information.

\item \textbf{period}: Should keep this at 0.0 unless one wishes to test accuracy for 
constant mesh at varying frequencies, or somehow the simulations show 
inaccuracies. This choice is ignored if a delta function is used as the source 
time function (in {\tt sourceparams.dat}), which should be the preferred option
unless wavefield snapshots (for propagation movies etc) are needed. 
\item \textbf{source file type}: Functional choices are {\tt separate} and {\tt cmtsolut}.
{\tt separate} invokes the generic source file {\tt sourceparams.dat}, and {\tt cmtsolut}
the {\tt CMTSOLUTION} format. See next section for details on these files. 
\item \textbf{receiver file type}: Functional choices are {\tt colatlon} and {\tt stations}.
{\tt colatlon} invokes the generic receiver file {\tt receivers.dat}, and {\tt stations}
the {\tt STATIONS} format. See two sections down for details on these files. 
\item \textbf{seismogram sampling rate}: Keeping this at 0.0 saves seismograms 
at the time step sampling rate. Choosing values larger than the time step may 
be useful to maintain a sampling rate equal to that of, e.g., data. Note though 
that keeping this value too high will result in crude seismograms which loose 
valuable frequency information.

\item \textbf{data and info output paths}: We suggest to keep those as defined; 
{\tt Data} contains all the large output such as seismograms and wavefields and all 
associated information. {\tt Info} contains additional information files, e.g. on 
processor-specific output and properties. 

\item \textbf{Save global snapshots}: Saves the displacement wavefield at discrete time 
intervalls for posteriori compilation into a wave propagation movie. These files are LARGE!
Only needed for visualization, understanding certain wavefield features, but should be 
generally turned off. The post-processing scripts turn these wavefields into a 3D global 
wave propagation movie if saved. NOTE: if this is set to {\tt .true.}, then a Dirac delta 
source doesn't make sense since the spurious high-frequency signals cannot be filtered 
a posteriori. 

\item \textbf{snapshot time interval}: Time in seconds between frames for the above snapshots 
of the propagating global wavefield. Obviously the output size depends directly on this intervall... 
For decent movies we suggest to keep this at 1-2 frames per seismic period. 

\item \textbf{save wavefields for kernels}: So long as the kernel code is not available, this 
is pretty useless since the resultant wavefields are subjected to an extensive amount of operations 
(rotations, coordinate transformations, frequency-domain transformations, convolutions, filtering)
to arrive at sensitivity kernels. Setting this to false ignores the rest of the choices in this output 
section for sensitivity kernels. CAUTION: These files are gigantic for high resolutions!

\item \textbf{Samples per period}: Should be at least 2 to fulfill Nyquist, but more 
are highly recommended for the purpose of picking time windows when 
computing static kernels, e.g. 8-10.

\item \textbf{source vicinity in wavefields}: Sensitivity kernels exhibit large values right around 
the source and receiver. Apart from ignoring this ({\tt igno}), 
we offer the possibility to simply {\tt mask} this region with smaller values. 
An analytical calculcation for homogeneous regions is in the making.

\item \textbf{starting/ending GLL point index}: Option to save only select GLL points per element for the 
kernel wavefields. Choosing 1,1 means only interior points, which maintains a smooth sampling (due to the 
GLL point clustering near the element boundaries) and avoids the peculiar and tricky issue of defining 
numerically accurate strains on discontinuities.

\item  \textbf{energy}: avoid this unless you are specifically interested in the kinetic,
potential and total energy of the solid, fluid and global domains since it 
involves a full evaluation of the stiffness matrix and is therefore quite 
CPU (and memory) intensive. 

\item  \textbf{homogeneous parameters}: This is mainly for testing purposes and should be 
turned to false otherwise. Only possible for purely solid models 
and only really useful for testing/debugging.

\item \textbf{homogeneous P-vel etc}: Populates the entire domain with this 
homogeneous media characterization if exclusively solid.

\item \textbf{analytical solution around source}: An analytical reference solution to test the 
accuracy of near-source radiation in a homogeneous model. Should only be on if this is relevant.

\item \textbf{add heterogeneity}: Turning this {\tt .true.} invokes an additional input file called {\tt inparam\_hetero}. 
This perturbs the background model in a specified region (defined by elements within a range of specified coordinates) 
by percentile changes to velocities $v_p$, $v_s$, and density
$\rho$. This part of the code is still under development but contains
a number of different types of heterogeneties, all of which are
defined in {\tt lateral\_heterogeneties.f90}. See more details in the separate section below. 
This should generally be set to false unless you specifically wish to analyse waveform effects due to such 
2.5-dimensional heterogeneties. 

\item Setting the next two lines to false (do mesh tests, save large test files) 
avoids extensive checks on the mesh, model, discontinuities etc. Upon a new mesh, this may be turned 
to true at least once (same after any source code changes), otherwise
it's fine to avoid these tests and large files.

\item A new (Nov. 2011) feature is the I/O using the netcdf
  standard. This is extremely beneficial for platform-independent
  binaries, direct access, reducing the number of written files, and
  is installed on every well-managed computer cluster. However,
  installing these libraries can be tedious. For this case, we have
  left the option to go with fortran binary. The decision is two-fold;
  the first one already taken in {\tt makemake.csh} since the makefile needs to
  know whether the libraries are installed. Even if installed, one can
  still turn the netcdf option off by putting binary here. 
\end{itemize}

\noindent \textbf{Source specification.}\\
Includes all parameters describing the earthquake (or other) source, which are: 
radiation patterns and magnitude, source location (depth, latitude, longitude), source-time function.
As of now, the code only computes point sources (finite faults are in development), 
and simply takes the closest grid point as the location. Mislocation errors are given in OUTPUT\_$<$dir$>$.\\

\noindent {\tt sourceparams.dat}
\begin{figure*}[htb]
\begin{center}
\includegraphics[scale=0.6]{sourceparams.eps}
\caption{\textit{{\tt sourceparams.dat}: Specifies source properties using its own format.}}
\end{center}
\end{figure*}

\begin{itemize}
\item  \textbf{magnitude}: The code runs in SI units, and the moment tensor is given 
in [N m], i.e. $10^{20}$ here for moments equal $10^{27}$ dyn cm in the {\tt CMTSOLUTION}
format further below. The moment-tensor elements here are given in global cartesian coordinates (Greenwich),
such that they are subjected to rotations if the source is not along the axis.
\item \textbf{excitation type}: earthquakes only contain mono-, di-, quadrupole radiation
\item \textbf{radiation pattern}: \\
monopole: {\tt explosion, mzz, mxx\_p\_myy, vertforce}\\
dipole: {\tt mxz,myz, yforce, xforce} \\
quadrupole: {\tt mxy, mxx\_m\_myy}\\
This is overdetermination, granted, but still a sensible check on whether 
one has the anticipated source defined - a VERY COMMON *source* of swiftly 
running wrong simulations! Consistency checks are included at run time.
\item \textbf{source depth}: NOTE: DEPTH in KM, NOT radius!
\item \textbf{source colatitude and longitude}: If the source is located away from the axis, 
the code rotates the entire source-receiver geometry such that the simulation runs with 
a corresponding source at the axis. This should be invisible to the user, but means that 
the raw seismogram output is not at the correct azimuth: This is accomodated in the post 
processing stage (see below). 
\item  \textbf{source time function}: Dirac Delta (single forces)/Heaviside (moment tensor)
should be used in all cases except for wavefield snapshots or specific tests. 
This instantaneous source detonates at the first time step of the seismogram 
sampling ({\tt t(seis\_it)}). Details below.
\end{itemize}

\noindent {\tt CMTSOLUTION}\\
\begin{figure*}[htb]
\begin{center}
\includegraphics[scale=0.6]{CMTSOLUTION.eps}
\caption{\textit{{\tt CMTSOLUTION}: Specifies source properties using the Harvard CMT format. Note that the source-time function
is added as the first string in the first line, if none is added, then a Dirac delta distribution is assumed.}}
\end{center}
\end{figure*}

\noindent If {\tt CMTSOLUTION} is specified, the moment tensor elements adhere to the spherical system 
(or what is called the ``local'' {\tt (x,y,z)}), i.e. is invariant to rotations. 
If the source is not at the pole, then the receiver coordinate system is automatically set to spherical, irrespective of {\tt inparam}:
this is so since in the cylindrical case, the rotated receiver system would need additional rotations, 
whereas the spherical system does not. The source time function in CMTSOLUTION can be specified by adding it as a string 
to the first line. If not specified, the default is the Dirac Delta distribution. Half duration corresponds to what we take 
as the dominant period in seconds, and time shift is ignored for now.
Depth is in kilometers, latitude and longitude in degrees, and the
moment tensor elements in [dyn cm], i.e. $10^7$ larger than the corresponding entries in {\tt sourceparams.dat}! \\

\noindent \textbf{Source-time function.}\\
%
\noindent 
Choices are between impulsive sources (Dirac Delta distribution,
Heaviside, see below) and smooth sources with a limited frequency band (Gaussian, Ricker,
derivative of Ricker and a wiggly wavelet with boxcar frequency 
spectrum (called {\tt heavis})), see the Figure for their shapes. 
The distributions start rupture at $t=t({\rm seis\_it})$ (the first time step counted in the seismogram 
sampling); the Gauss, Ricker and derivatives have the wavelets
centered at $t=1.5 \,T_0$, and the {\tt heavis} function is shifted by $300 s$ due to its very long wavelength components. These time 
shifts are recorded and saved into post processing, but it is crucial that users are aware of these shifts when comparing to 
data or other methods. \\
\begin{figure*}[htb]
\begin{center}
\includegraphics[scale=0.45]{stfs.eps}
\caption{\textit{The smooth source time functions for a period of $T_0=20$s. Note the shifted center for each function 
($1.5 T_0$ for all except the wiggly wavelet which is shifted by $300s$). The amplitude includes the scalar moment of the source. }}
\end{center}
\end{figure*}

\begin{figure*}[htb]
\begin{center}
\includegraphics[scale=0.45]{power_spectrum_stf.eps}
\caption{\textit{Power spectra of the source time functions from the previous Figure.}}
\end{center}
\end{figure*}

\noindent \textbf{Discrete Dirac distributions.}\\
Hidden in the code are several crucial decisions upon how to
approximate Dirac and Heaviside distributions with discrete and
possibly heavily downsampled time series. These aspects are dealt with
in subroutine {\tt compute\_numerical\_parameters} in module {\tt
  parameters.f90}, and subroutine {\tt delta\_src} in {\tt module
  sources.f90}.
In summary, a triangular Dirac is taken if there is no downsampling,
leading to perfect reconstruction of the Dirac properties. If
seismograms or wavefields are downsampled (as specified in {\tt
  inparam}), then one needs to revert to "smooth" functions that
replicate Dirac properties. Amongst a choice of functions, 
we set the default to a tight Gaussian scaled to honor the Dirac
integral unity,
whose half width depends on the time step of the SEM, the
seismogram/wavefield sampling rate, and mesh-based period. In
principle, we observe acceptable reconstruction of Green's functions 
if more than 10 points per (mesh) period are used. The discrete Dirac
functions are shifted such that they (1) start at zero smoothly, (2)
are exactly sampled by all the sampling rates (SEM time step,
seismogram, wavefield sampling rates). The crucial parameters for
these shifts are saved for the post processing step and dealt with
there. In principle, users should not need to worry about these
issues, but be aware of the sensitivity of results upon these
specifications. Heaviside approximations are simply numerically
integrated from the Dirac distributions.\\

\textbf{WARNING:} The implementation of approximate/smooth Dirac
distributions has not yet been implemented for the symplectic schemes,
as these require different temporal sampling and hence more layers of
if's, dont's, and special cases in the definition of these approximate Diracs.

\noindent \textbf{Receiver specification.}\\
At this point two receiver file formats are accepted, both of which specify the (co-)latitude and longitude of the
receivers. The code simply finds the closest grid point which, due to the
crustal discretizaton, is usually of negligible difference. Check {\tt OUTPUT\_$<$dir$>$} to see how mislocated they are.
Note that specifying a non-zero longitude is essentially just saved to output and loaded in the 
post-processing stage. This is due to the 2D nature of the computational domain, and the 3D location is analytically reconstructed 
a posteriori. Also note that these are \textit{globally fixed coordinates}: If the source is not meant to be beneath the North pole 
(as specified in the source input file), then the combined source-receiver geometry is rotated within the code such that the source 
coincides with the North pole as necessary for the simulation. 
To retrieve the response at the proper location on the 2D surface, 
this involves component rotations and is taken care of in the post processing stage. In the current 
workflow, is intended that the user can safely ignore these cumbersome operations so long as one appropriately applies 
the post processing operations. \\

\noindent {\tt receivers.dat}\\
If {\tt colatlon} is specified in {\tt inparam} (see above), then a file called {\tt receivers.dat} in the source-code directory 
is needed. This file contains a simple list of co-latitudes and longitudes. 
The first line denotes the number of receivers to follow, i.e. length of file minus one:\\
{\tt <number of receivers>\\
<rec. 1 colatitude [deg]> <rec 1 longitude [deg]>\\
<rec. 2 colatitude [deg]> <rec 2 longitude [deg]>}\\

\noindent 
A script called {\tt create\_receiver\_file.bash} is provided in the {\tt UTILS} directory to generate such files with constant 
spacing. Choose number of receivers as an argument and the resultant file will spread receivers along the colatitude between 0 and 
180 degrees.\\

\noindent {\tt STATIONS}\\
This choice requires a file called {\tt STATIONS} in the source-code directory. As an example,
in the {\tt UTILS} directory you will find a file named {\tt STATIONS\_all\_15aug2008\_cleaned} 
with more than 3000 actual seismometer locations.
The file structure is generic, but do note that this format uses the latitude rather than colatitude 
(in opposite to the {\tt receivers.dat} format). 
Elevation and burial are ignored for the time being. Station names are transfered to the 
output seismograms in {\tt Data}, see section on output for more details.

\subsection{Adding lateral heterogeneities}
\begin{figure*}[htb]
\begin{center}
\includegraphics[scale=0.5]{inparam_hetero.eps}
\caption{\textit{{\tt inparam\_hetero}: defines the region of lateral heterogeneities and medium variations.}}
\end{center}
\end{figure*}
\noindent 
\textbf{NOTE:} This section has been heavily expanded and used most
recently (fall 2011), but not yet tested in full extent. Everything
concerning lateral heterogenities is confined to module {\tt
  lateral\_heterogeneities}.\\

If the boolean {\tt add\_heterogeneities} in {\tt inparam} is set to true, then a file named {\tt inparam\_hetero.dat} is required 
which specifies (1) a region within which heterogeneities are added and (2) the perturbations of the model parameters in 
percent with respect to the background model. The example in the Figure describes a boxcar atop the core-mantle boundary
in which variations are constant, resembling simplistic models of an Ultra-Low Velocity Zone. The heterogeneity is added 
to the background model in the module {\tt get\_model.f90}, routine {\tt read\_model.f90}. The simple case described 
here is no inherent limitation but simply the first step into the direction of adding heterogeneities. More complex 
shapes can be defined as well as non-constant model perturbations, but this involves modifying the input file and the routine 
in which this takes place. \\

\noindent \textbf{IMPORTANT NOTE}: AXISEM is based upon the assumption of spherically symmetric background models to 
rely on the specific radiation patters that allow for a dimensional collapse to a 2D computational domain. 
If one adds lateral heterogeneities as above, the code simulates wavefields as if they penetrate a ring-like, tube structure 
that is invariant along the azimuth. This is often called 2.5-dimensional modeling. Thus, if one is only interested in 
the relative waveform impact of heterogeneous structures along the source-receiver plane, and in addition simulates 
at sufficiently high frequencies, then only a fraction of the ring structure will be ``seen'' by the transient wave. Although 
this has been common practice for many studies with various other codes beforehand, it 
is important to keep these topics in mind when interpreting results from lateral heterogeneities. Note also that while 
the structure is 1D (spherically symmetric) and, if the above is applied, locally 2.5D, the resultant wavefield is always 
the full 3D response to these background models.

\subsection{Running the solver}
We use the following script to submit a parallel job from the source-code directory:\\
{\tt ./submit.csh <run\_directory> <additional arguments>}\\

\noindent {\tt submit.csh}\\ 
Check{\tt ./submit.csh -h} for argument options. 
This script performs a number of crucial operations for the code to run, e.g. adheres to 
the source and receiver specifications, creates subdirectories, copies relevant files into them,
and can submit a full moment (4 simulations) simultaneously.
The {\tt <run\_directory>} can be a global or local path. In most cases, one should keep the source code in a 
safe (backed up) location, whereas {\tt <run\_directory>} may be a faster/larger scratch system. 
By experience, it is good practice to include all crucial parameters into the name 
of the run, specifically: mesh resolution, background model, number of processors, 
moment tensor type, source depth, source time function, and output choices (e.g. wavefield snapshots, kernel wavefields), e.g.
{\tt PREM\_20s\_NP2\_Mzz\_100km\_GAUSS0\_SNAPS}. 
The script generates {\tt param\_sum\_seis} in the run directory containing the number of these simulations 
and their respective locations. 

\subsection{Computational aspects}
\noindent A monopole simulation for a dominant period of 10 seconds on 2 processors and 
one-hour long seismograms (without saving large wavefields) should take no longer than 
an hour, i.e. \textbf{real-time on a laptop} (see Table 1). 
It is important to recognize that a simulation at half the dominant period compared to a previous simulation 
takes about 8 times longer if seismogram length is fixed: The mesh is about 4 times larger, and the time step about twice as small.
Note that monople source types run faster and require less RAM than dipoles and quadrupoles.\\

\begin{table*}[htb] 
\begin{minipage}{150mm}
\caption{ \textit{RAM and CPU requirements for simulations at dominant period 10s, PREM}.}
\label{apptable:matrix_op}
\vspace*{.2cm}
\begin{tabular}{@{}cccc}
nproc & RAM/CPU [MB] & seismogram length [s]& CPU time [s]\\
\hline\\
1 & 380 & 1000 & 1717\\
2 & 210 & 1000 & 1076
\end{tabular}
\end{minipage}
\end{table*}

\noindent \textbf{Performance.} Our experience favors the Intel compiler using optimization flags {\tt -O4 -xHOST}, but 
Pathscale (not that they offer the compiler open-source now) performs very well too.
Performance can be severely hampered by writing to a different/slow file system, running a parallel job on nodes that 
are partly busy with other tasks, reaching the RAM limits (or swapping). In terms of input choices, saving wavefields 
for snapshots or sensitivity kernels take up significant portions of the CPU time as well. The Theory section contains 
some more details on performance checks.\\

\noindent \textbf{Scaling.} The code scales very well: both weak and strong scaling have showed above 90\%. 
This is due to the exact load balance and minimal number of processor neighbors (two) as well as the asynchronous 
message passing. The Solver output contains some run-time information at the end, in which the time spent in 
crucial routines (assembly, time loop, saving wavefields) is calculated. 

\subsection{Code structure}
\noindent {\tt main.f90} is  the wrapper routine, and most of the relevant seismologically relevant 
routines are called from {\tt time\_evol\_wave.f90}. The overall code contains $\sim$ 200 potentially 
job-terminating if-statements which test a variety of generic issues prior 
to the time loop, and as such it should be fairly robust once the time loop is entered. \\

\noindent \textbf{Time loop.} 
Several time schemes are included, most importantly the 2nd-order Newmark scheme and a symplectic 4th-order 
scheme. See next section for details.
Most of the heavy floating-point-operation-intensive number crunching happens in unrolled\_loops.f90, which 
is a routine optimized for faster cache access following Deville et al. 2003. 

\subsection{Choice of appropriate time schemes}
In most other spectral-element codes, the 2nd-order Newmark scheme is 
the main option for time schemes. It performs fairly well for a certain parameter regime, so long 
as the time step is chosen with care. To obtain small dispersion errors for large 
propagation distances (e.g. over more than 100 wavelengths), higher order symplectic 
schemes are more effective. See the Theory section or Ref (5) for more details.\\

\noindent \textbf{Newmark scheme.}
 The conventional explicit, acceleration-driven Newmark scheme of 2nd order.
(e.g. Chaljub \& Valette, 2004). The mass matrix is diagonal; we only store 
 its pre-assembled inverse at the stage of the time loop.
 Explicit axial masking follows ref (2).
 Note that the ordering (starting inside the fluid) is crucial such that no 
 iterations for the boundary terms are necessary.
 Also note that our definition of the fluid potential is different from 
 Chaljub \& Valette and the code SPECFEM by an inverse density factor.
 This is the correct choice for our case of non-gravitating Earth models, 
 but shall be altered once gravity is taken into account.\\

\noindent \textbf{Symplectic schemes.}
Several higher-order time schemes are included that solve the 
coupled solid-fluid system of temporal ODE's
using symplectic time integration schemes of 4th or 6th order.
The time step can be chosen 1.5 times larger than in Newmark, resulting 
in CPU times about 2.5 times longer than Newmark, but considerably more 
accurate. Consult Ampuero \& Nissen-Meyer (2011) for examples of when 
this choice should be more appropriate. Generally, for long propagation 
distances (say, $>$ 100 wavelengths), it is worthwhile considering this scheme. 
Note also that symplectic schemes actually occupy \textit{less memory}
at run time.\\

\textbf{WARNING:} The implementation of approximate/smooth Dirac
distributions has not yet been implemented for the symplectic schemes,
as these require different temporal sampling and hence more layers of
if's, dont's, and special cases in the definition of these approximate Diracs.

\subsection{At run-time} 
Everything related to a run is located in the run directory: {\tt cd <run\_directory>; ls}\\
\noindent You should see an executable, input and 
header files, a file called {\tt OUTPUT\_$<$run\_directory$>$} (the file name may replace slashes 
with underscores if global paths are used) and files {\tt output\_proc00*.dat},
as well as time stamps eventually (the pre-timeloop CPU time may take a few 
minutes). Time stamps are written every estimated 1\% of the total runtime. 
Min/max displacement values are recorded and may be a good starting point in 
deciphering a potential blow-up of the simulation. Note that this happens if the time step 
is too large, whereas inaccurate seismograms (containing high-frequency wiggles) happens 
if the grid spacing is too large.

\subsection{Output}
The Solver produces vast amounts of various types of output, 
both in the {\tt Info} and {\tt Data} directories (as specified in {\tt inparam}. We will 
only cover the most relevant ones here, most other output is self-explanatory 
or well commented in the code, or irrelevant ;-)
General information on the run can be found in {\tt simulation.info}, besides 
the generic standard {\tt OUTPUT\_<run\_directory>} and the processor-specific files
{\tt output\_proc00*.dat}. Google-earth kml files that describe the input source-receiver
geometry and, if necessary, the rotated one used in the code are provided in {\tt Info/src\_rec*.kml}.
The source-time function is in {\tt Data/stf.dat}. All of the following output is located in {\tt Data}.\\

\noindent \textbf{Seismograms}: \\
Default output are seismograms at the epicenter, hypocenter,
equator, and antipode. These are organized as (time, displacement component), and 
called e.g. {\tt seisepicenter1.dat} for the first component.
All other seismograms based on the chosen input 
({\tt receivers.dat} or {\tt STATIONS}) contain the displacement 
components only (no time column). Seismograms from {\tt receivers.dat} are called 
{\tt recfile\_00*\_disp.dat} and {\tt recfile\_00*\_velo.dat} for displacement and velocity, respectively.
Seismograms from {\tt STATIONS} contain the station name. Note that these seismograms 
are the collapsed-dimension equivalent, so in principle we advise to not touch them but 
rather proceed with post processing to properly rotate them to the correct location before 
any further analysis.\\

\noindent \textbf{Wavefield snapshots}:\\
Displacement snapshots are saved into files {\tt snap\_<proc number>\_<time sample>.dat}
and saved across the entire mesh, additionally a corresponding mesh file is written.
These wavefields are loaded into the post processing procedure and converted to 3D 
movies.\\

\noindent \textbf{Kernel wavefields and seismograms}: \\
Similar to snapshots, but the strain tensor needs 
to be computed on the fly first, see {\tt time\_evol\_wave.f90}. 
At this point, we offer two end-member versions of dumping the fields  
to eventually calculate waveform kernels:

\begin{enumerate}
\item \textit{displ\_only} dumps a minimal amount and requires extensive 
 post-processing when calculating the kernels, but optimizes the SEM 
 simulation in terms of memory, storage amount and CPU time.
Note that this method IS ONLY POSSIBLE IF ENTIRE SEM MESH IS DUMPED.
This means minimal permanent storage, minimal run-time memory, minimal CPU time, 
but extensive post-processing (need to compute strain tensor).

\item \textit{fullfields} computes the entire strain tensor and velocity 
 field on-the-fly, resulting in more output (9 rather than 6 fields), 
 more memory and CPU time during the SEM, but no post-processing necessary. 
This means maximal permanent storage, maximal run-time memory, maximal CPU time, 
but no post-processeing necessary as these are the fields that 
constitute density and elastic kernels.
Any kind of spatial distribution can be dumped, meaning in the long run 
this should be the more effective choice, and we recommend using this option.
\end{enumerate}

\noindent \textbf{Energy}: \\
Time series for total, kinetic, and potential energy of global,
solid and fluid domains.

\subsection{Cautions}
The code should NOT be compiled with the {\tt -r8} flag (automatic double precision), since the vtk output 
assumes single precision. All sensitive parameters (e.g. coordinates) are kept at double precision when needed internally anyway...

\subsection{Issues}
For 1.5 elements per wavelength (specified in {\tt inparam\_mesh}), there are still some spurious wiggles 
most notably at zero epicentral distance (the axis). Might come from the fact that meshing is not appropriately accomodating 
the Gauss-Lobatto-Jacobi basis near the axis, check {\tt discont\_meshing.f90} in the mesher. 
Choosing a higher period or denser mesh takes care of this for the time being.

\subsection{Solver To Do}
\noindent Physics: Anisotropy, attenuation, oceans, fully fluid sphere, (gravity, rotation)\\

\noindent Kernel output: incorporate boundary kernels, wavelet compression or similar, writing wavefield snapshots into
frequency domain\\

\noindent Multiple source flexibility: Option to run 4 simulations for a complete moment tensor 
consecutively without changing input parameters, in combination with 
actual receiver and source coordinates (coordinate and component rotations) for seismograms and snapshot wavefield. 
Also, the option to submit one parallel job distributed across 4*nproc processors to do the entire moment tensor and 
sum on-the-fly. Finally, the option to run multiple source depths in one parallel job submission: This has the advantage 
that only the dynamic wavefields and source term differ from one source to the other, i.e. one saves a lot of total memory 
compared to separate runs. Vision: One job submission (hundreds of processors) for the entire database...\\

\noindent Embedded multiple simulations: Parts have been started (see the loop in {\tt main.f90}), but this is far from finished. 
  Two options should be done: \\
        1) Sequential 4 simulations for the full moment tensor, but invoked from within the code, not in the submit script\\
        2) Parallel runs for all 4 simulations, where all systems (mono/di/quadrupole are loaded simultaneously). Disadvantage that 
            monopole runs are faster... so any internal seismogram summation will lead to idle processes (unless completely parallelized such that each processor owns parts of each radiation type). This is a bit of recoding...\\

-netcdf input (mesh database) and output (wavefield snapshots and kernel wavefields)\\

- GPU version\\

- local timestepping


\newpage
\section{Post processing}
Once the simulation is done, you'll have to run post processing in the
run directory. This code performs the following 
crucial tasks:
\begin{itemize}
\item Load seismograms
\item Convolve seismograms with newly defined source time function and seismic period
\item Compute seismogram at correct azimuth
\item Sum individual seismograms to full moment tensor (if applicable)
\item Rotate seismogram components to new coordinate system
\item If available, plots seismograms using {\tt gnuplot} in pdf and
  gif formats
\item shifts seimograms such that the source-time function coincides
  with time zero (either starting at negative time or at zero)
\item Constructs a google-earth kml file that plots source-receiver geometry on 3D surface, source information, 
and seismograms as images
\item If applicable, loads wavefield snapshots and computes 3D wave propagation for wavefields upon full
moment tensor source, using the cubed sphere on 2 surface (top and bottom) and two cross-sections (``cakepiece'') 
inside the 3D sphere
\item If available, computes theoretical traveltimes for the source-receiver configuration using TauP
\item Offers matlab script to plot record sections of all seismograms
\end{itemize}

\noindent Before running this script, it is useful to check the parameter files for post processing that were automatically 
generated by the Solver based on assumptions about what you might be interested in:\\

\begin{figure*}[htb]
\begin{center}
\includegraphics[scale=0.65]{param_post_processing.eps}
\caption{\textit{{\tt param\_post\_processing}: Automatically generated input file for post processing.}}
\end{center}
\end{figure*}

\noindent {\tt param\_post\_processing}\\
This file is automatically generated by the Solver but should always be checked before running the post processing. 
If a full moment is submitted, then {\tt param\_post\_processing} is in the actual run subdirectory (usually called {\tt MZZ} etc). 
{\tt post\_processing.csh} takes care of copying this into the main directory, 
but if you wish to edit it before processing, then copy one of them beforehand 
(e.g. {\tt cp MZZ/param\_post\_processing .}) and edit locally.\\

\noindent {\tt param\_snaps}\\
\begin{figure*}[htb]
\begin{center}
\includegraphics[scale=0.65]{param_snaps.eps}
\caption{\textit{{\tt param\_snaps}: Input for {\tt post\_processing.f90} created by the Solver. These parameters 
control the geometry of the slices/surfaces of the 3D sphere upon which wavefields are projected.  }}
\end{center}
\end{figure*}

\noindent 
If wavefield snapshots were saved in the Solver, an input file named {\tt param\_snaps} was created.
The 3D sphere (where cross sections are taken and rotated from the 2D mesh, and surfaces constructed 
using a cubed-sphere projection) consists of maximally 4 parts: 
two cross sections $(r,\theta)$ at the two $phi$ values specified in {\tt param\_snaps}, 
and 2 surfaces, specified by their radii. These surfaces contain a lot of points and wave propagation is not 
all that interesting, so if not needed for visual pleasure we suggest to avoid at least the top surface for a computational
shortcut. The snap number can also be controlled via starting, ending snap as well as skipping factor. \\

\noindent \textbf{Run post processing}\\
Inside the main run directory: {\tt ./post\_processing.csh}.\\ See {\tt ./post\_processing.csh -h} for help/info on the 
argument options. The main part of post processing is done by {\tt post\_processing.f90}, which 
has been compiled by {\tt submit.csh}. Make sure the executable {\tt xpost\_processing} is located in the run directoty.
All output from post processing is saved into the directory specified in param\_post\_processing. 
In case of a moment summation, these directories exist locally (i..e in directories MZZ, etc) and in the run directory, where 
the main directory folder contains the summed results. Standard output of the script is written to screen with information 
on the tasks and resultant output files, and standard output from the main routine {\tt post\_postprocessing.f90} is 
written to {\tt OUTPUT\_postprocessing}. The script creates up to 4 directories: 
{\tt GRAPHICS, SEISMOGRAMS, TAUP, SNAPS}. 
Content is self-explanatory, described in the standard output, and below.
\newpage
\subsection{Post processing output}

\noindent \textbf{Processed seismograms}\\
Processed seismograms (i.e. at the correct surface location, possibly rotated components, convolved, summed to a full 
moment tensor) are located in subfolder {\tt SEISMOGRAMS}.\\

\noindent \textbf{Google earth source-receiver geometry}\\
\begin{figure*}[htb]
\begin{center}
\includegraphics[scale=0.65]{googleearth_src_rec.eps}
\caption{\textit{The kml file output from post processing. It contains the rotated, original source-receiver geometry. Mouse-clicks 
on earthquake location provide source information, mouse-clicks on receiver pins receiver location information and graphics 
of the local seismograms.}}
\end{center}
\end{figure*}

\noindent \textbf{Wavefield snapshots in 3D}.\\
The resultant vtk files are saved into {\tt Data\_Postprocessing/SNAPS/snap*vtk} and can be animated to a movie 
with paraview. Each processor dumps its own file.\\

\noindent \textbf{We have tested:}\\
- rotations from cylindrical to spherical both in the solver and post processing\\
- moment tensor summation to explosion source using moment versus direct inclusion in solver\\
-  compatibility of the two source and receiver file types (including cross-usage)\\
- summation of wavefield snapshots\\
- post processed convolution versus bandlimited stf in solver\\

\noindent \textbf{To Do}:\\
- turn strain wavefields into frequency domain for kerner\\
- rewrite scripts in Python\\
- parallelize snapshot creation\\
- netcdf seismogram databases

\section{Example workflows}
- this is under construction, but in directory EXAMPLES, you'll find
the input files for normal-mode summation code Mineos
(geodynamics.org), against which we have performed some
benchmarks. Caution: We have not managed to create a mode catalog that
is complete, in other words the reference solution is not to be seen
as precise in a quantitative way. Some pdfs in the directory show our
fit. It is useful to re-run such tests when first using AXISEM.

%\newpage
\section{Kernel calculations}
Not included yet at this point... stay tuned!\\

% \noindent The philosophy of this 'scattering-integral' approach to sensitivity kernels 
% is a once-and-for-all calculation of a database for a given background model,
% i.e. including all potential earthquake depths. Everything else related to 
% data, measurements, inversion parameters and observables, time windows, etc 
% is done at the stage of inversion. At this time, output is limited to 
% saving snapshots of the entire 2D mesh or a (contiguous) selection of grid points 
% within elements. Under development is a more 
% sophisticated method to either sample more sparsely, output into a wavelet 
% basis, and straight into frequency domain. 

% In addition to using such sensitivity kernels for inversions, 
% these can be used for Born modeling such that for any given 3D tomographic earth model, 
% it is only a matter of a volumetric integration to determine the seismogram upon such tomographic 
% models. This is not done yet either.

% \subsection{Kerner installation and compilation}
% \subsection{Running the Kerner}
% \subsection{Kerner input}
% \subsection{Kerner computational aspects}
% \subsection{Kerner code structure}
% \subsection{Kerner output}
% \subsection{Kerner post-processing}
% \subsection{Kerner To Do}
\newpage
\section{To Do General}
\subsection{Mesher}
\textbf{Top priority:} Parallelization, 32 processor domain decomposition, netcdf databases\\

\noindent\textbf{Medium priority:} Bending elements for sharp lateral
discontinuities\\

\noindent\textbf{Low priority:} Commenting, cleaning , ellipticity, strongly deformed sphere
(exostars)

\subsection{Solver}
\noindent\textbf{Top priority:} 
\begin{itemize}
\item Attenuation, transverse anisotropy
\item benchmark heterogeneities
\item finite faults, multiple source simulations (4 Mij,
multiple depths, sequential and parallel)
\item dump fields in theta slices, dump fields in frequency domain,
  netcdf, compression
\item Link input to NDLB
\end{itemize}

\noindent\textbf{Medium priority:}
\begin{itemize}
\item Oceans, fluid sphere (surface boundary condition),
  inner-core anisotropy
\item GPU version, local high-order timestepping
\item Incorporate boundary-kernel dumps
\item non-blocking MPI, parallelization benchmark

\end{itemize}

\noindent\textbf{Low priority:}
Gravity, python rewrite of submit.csh

\subsection{Post-processing}
\noindent\textbf{Top priority:} 
\begin{itemize}
\item Benchmark snapshot movies
\item Resampling of timeseries
\item rewrite wavefields in frequency domain
\item finite fault summation
\item extraction from synthetics database (VERCE, IRIS)
\item netcdf, streamline input
\end{itemize}

\noindent\textbf{Medium priority:}
Parallelize snapshot dumps (and other operations)\\

\noindent\textbf{Low priority:} python version

% \subsection{Kerner}
% \noindent\textbf{Top priority:} 
% \begin{itemize}
% \item Quadrature
% \item rotation of kernels from/to actual coordinates
% \item Code efficiency
% \item Benchmarks
% \item misfits
% \item summation for Mij
% \item event-based code (shortcuts for multiple receivers), $G_{ij}$-based code
% \item anisotropy kernels
% \item Link input/output to NDLB
% \item incorporate boundary kernel code
% \end{itemize}

%\noindent\textbf{Medium priority:}
%Born modeling, GPU, On-the-fly kernel generation (fwd wavefield \&
%kernel)\\

%\noindent\textbf{Low priority:}
%Attenuation kernels

%\subsection{Miscellaneous}
%Manual, generation of synthetics database, code release (CIG, VERCE, QUEST)

%\newpage
%\input{theoretical_foundations}

\end{document}
